\documentclass[12pt,a4]{article}
\usepackage[english]{babel}
\usepackage[utf8]{inputenc}
\usepackage[T1]{fontenc}
\usepackage{geometry}
\usepackage{float}
\geometry{
	a4paper,
	left=20mm,
	right=20mm,
	top=20mm,
	bottom=20mm,
}
% Useful packages
\usepackage{amsmath}
\usepackage{ amssymb }
\usepackage{bm}
\usepackage{graphicx}
\usepackage[colorlinks=true, allcolors=blue]{hyperref}

\begin{document}
\section{The system}
The vessel is modelled by
\begin{align}
	\bm{M}_{RB}\bm{\dot{\nu}} + \bm{M}_{A}\bm{\dot{\nu}_r} + \bm{C}_{RB}(\bm{\nu})\bm{\nu} + \bm{C}_{A}(\bm{\nu}_r)\bm{\nu}_r
	+ \bm{D}(\bm{\nu}_r)\bm{\nu}_r +\bm{G}\bm{\eta} & = \bm{\tau} + \bm{w}(t) \\
	\bm{\dot{\eta}}                                 & = \bm{J}(\eta)\bm{\nu}
\end{align}
where
\begin{align}
	\bm{M}_{RB} & = \begin{bmatrix}
		m & 0 & 0     \\
		0 & m & 0     \\
		0 & 0 & I_{z}
	\end{bmatrix} &
	\bm{M}_{A}  & = \begin{bmatrix}
		-X_{\dot{u}} & 0            & 0            \\
		0            & -Y_{\dot{v}} & -Y_{\dot{r}} \\
		0            & -N_{\dot{v}} & -N_{\dot{r}}
	\end{bmatrix}
\end{align}


\begin{align}
	\bm{C}_{RB} & = \begin{bmatrix}
		0           & 0   & -m(x_g r +v) \\
		0           & 0   & mu           \\
		m(x_g r +v) & -mu & 0
	\end{bmatrix} &
	\bm{C}_{A}  & = \begin{bmatrix}
		0                                                        & 0               & Y_{\dot{v}} v_r + \frac{1}{2}(N_{\dot{v}} Y_{\dot{r}})r \\
		0                                                        & 0               & -X_{\dot{u}} u_r                                        \\
		-Y_{\dot{v}} v_r + \frac{1}{2}(N_{\dot{v}} Y_{\dot{r}})r & X_{\dot{u}} u_r & 0
	\end{bmatrix}
\end{align}

\begin{align}
	\bm{D}_{L}  & = \begin{bmatrix}
		-X_u & 0    & 0    \\
		0    & -Y_v & -Y_r \\
		0    & -N_v & -N_r
	\end{bmatrix} &
	\bm{D}_{NL} & = \begin{bmatrix}
		-X_{|u|u}|u| - X_{uuu}u^2 & 0                           & 0                           \\
		0                         & -Y_{|v|v}|v_r| -Y_{|r|v}|r| & -Y_{|v|r}|v_r| -Y_{|r|r}|r| \\
		0                         & -N_{|v|v}|v_r| -N_{|r|v}|r| & -N_{|v|r}|v_r| -N_{|r|r}|r|
	\end{bmatrix}
\end{align}

Using property 8.1 of [Fossen] we can define the dynamics using only the water relative velocity $\bm{\nu}_r$
\begin{align}
	\label{eq:SysEq} \bm{M}\bm{\dot{\nu}}_r + \bm{C}(\bm{\nu}_r)\bm{\nu}_r +\bm{D}(\bm{\nu}_r)\bm{\nu}_r & = \bm{\tau}                           \\
	\bm{\dot{\eta}}                                                                                      & = \bm{J}(\eta)\bm{\nu}_r + \bm{\nu}_c
\end{align}
Where $\bm{\nu}_c$ is the velocity of the water current in global coordinates. And
\begin{align}
	\bm{M}             & = \bm{M}_{RB} + \bm{M}_{A}                         \\
	\bm{C}(\bm{\nu}_r) & = \bm{C}_{RB}(\bm{\nu}_r) + \bm{C}_{A}(\bm{\nu}_r)
\end{align}
Additionally $\bm{G}\bm{\eta} = \bm{0}$ as there exists no restoring forces in surge, sway and yaw.
Isolating the acceleration $\bm{\dot{\nu}}_r$ in (\ref{eq:SysEq}) we have
\begin{align}
	\bm{\dot{\nu}}_r & = \bm{M}^{-1} (\bm{\tau} -\bm{C}(\bm{\nu}_r)\bm{\nu}_r -\bm{D}(\bm{\nu}_r)\bm{\nu}_r ) \\
	\bm{\dot{\eta}}  & = \bm{J}(\eta)\bm{\nu}_r + \bm{\nu}_c
\end{align}
Assuming that $\bm{M}$ and $\bm{C}$ as well as the linear damping coefficients $X_u$ and $N_r$ are known. The system equation can be split into a known part and a unknown or uncertain part. This can be written as
\begin{align}
	\bm{\dot{\nu}}_r & = \bm{M}^{-1} (\bm{\tau} -\bm{C}(\bm{\nu}_r)\bm{\nu}_r - \bm{d} \bm{\nu}_r + \Phi(\bm{\nu}_r )\bm{\vartheta}) \\
	\bm{\dot{\eta}}  & = \bm{J}(\eta)\bm{\nu}_r + \bm{\nu}_c
\end{align}
Where
\begin{equation}
	\bm{d} = \begin{bmatrix}
		X_u & 0 & 0   \\
		0   & 0 & 0   \\
		0   & 0 & N_r
	\end{bmatrix}
\end{equation}
And the uncertain part is described by
\begin{align}
	\Phi(\bm{\nu}_r)^T & = \begin{bmatrix}
		|u|u & 0 & 0 \\ u^3 & 0 & 0 \\ 0& v &0 \\ 0& r &0 \\ 0&|v|v&0  \\ 0&|r|v&0  \\ 0&|v|r&0  \\ 0&|r|r&0  \\ 0&0& v \\ 0&0& |v|v \\ 0&0&|r|v \\ 0&0&|v|r \\ 0&0&|r|r
	\end{bmatrix}
	                   &
	\bm{\vartheta}     & = \begin{bmatrix}
		X_{|u|u} \\ X_{uuu} \\ Y_v \\ Y_r \\ Y_{|v|v} \\ Y_{|r|v} \\ Y_{|v|r} \\ Y_{|r|r} \\ N_v \\ N_{|v|v} \\ N_{|r|v} \\ N_{|v|r} \\ N_{|r|r}
	\end{bmatrix}
\end{align}
Lastly the known part is collected in a function $\bm{f}(\bm{\nu}_r)$.
\begin{align}
	\bm{\dot{\nu}}_r & = \bm{M}^{-1} (-\bm{C}(\bm{\nu}_r)\bm{\nu}_r -\bm{d} \bm{\nu}_r) + \bm{M}^{-1} (\bm{\tau} + \Phi(\bm{\nu}_r )\bm{\vartheta}) \\
	                 & =  \bm{f}(\bm{\nu}_r) + \bm{M}^{-1} (\bm{\tau} + \Phi(\bm{\nu}_r )\bm{\vartheta})                                            \\
\end{align}
The system is now in a form where the known part $\bm{f}(\bm{\nu}_r)$, the uncertain part $\Phi(\bm{\nu}_r )\bm{\vartheta}$ and the input $\bm{\tau}$
are clear to distinguish from one another.

\section{Adaptive Velocity Control}
\subsection{Control Design}
The control objective is to minimize $\tilde{\nu}(t) \triangleq \nu(t) - r_{\nu}(t)$.
That is to minimize the difference (error) between the velocity of the vessel $\nu(t)$ and the reference velocity $r_{\nu}(t)$.
A new variable $\bm{z}$ is introduced to describe this difference.

\begin{equation}
	\bm{z} \triangleq \nu - r_{\nu}
\end{equation}
Choosing the Lyapunov function

\begin{equation}
	V \triangleq \frac{1}{2}\bm{z}^T\bm{z} + \frac{1}{2}\bm{\tilde{\vartheta}}^T\bm{\Gamma}^{-1}\bm{\tilde{\vartheta}}
\end{equation}
Which has the derivative

\begin{align}
	\dot{V} & = \bm{z}^T\dot{\bm{z}}
	+ \bm{\tilde{\vartheta}} ^T\bm{\Gamma}^{-1}\bm{\dot{\tilde{\vartheta}}}                                                                         \\
	        & = \bm{z}^T\left(\bm{\dot{\nu}}_r  - \dot{r_{\nu}}\right)
	+ \bm{\tilde{\vartheta}} ^T\bm{\Gamma}^{-1}\bm{\dot{\hat{\vartheta}}}                                                                           \\
	        & = \bm{z}^T\left(\bm{f}(\bm{\nu}_r) + \bm{M}^{-1}\left(\bm{\tau} + \Phi(\bm{\nu}_r )\bm{\vartheta}\right)  - \dot{r_{\nu}}\right)
	+ \bm{\tilde{\vartheta}} ^T\bm{\Gamma}^{-1}\bm{\dot{\hat{\vartheta}}}                                                                           \\
	        & = \bm{z}^T\left(\bm{f}(\bm{\nu}_r) + \bm{M}^{-1}\left(\bm{\tau} + \Phi(\bm{\nu}_r )\bm{\hat{\vartheta}}\right) - \dot{r_{\nu}}\right)
	+ \bm{z}^T\bm{M}^{-1} \Phi(\bm{\nu}_r )\bm{\tilde{\vartheta}}
	+ \bm{\tilde{\vartheta}} ^T\bm{\Gamma}^{-1}\bm{\dot{\hat{\vartheta}}}                                                                           \\
	        & = \bm{z}^T\left(\bm{f}(\bm{\nu}_r) + \bm{M}^{-1}\left(\bm{\tau} + \Phi(\bm{\nu}_r )\bm{\hat{\vartheta}}\right) - \dot{r_{\nu}}\right)
	+ \bm{\tilde{\vartheta}}^T \left(\bm{\Gamma}^{-1}\bm{\dot{\hat{\vartheta}}} +\Phi(\bm{\nu}_r )^T \bm{M}^{-T} \bm{z} \right)                     \\
\end{align}
It is seen that choosing the control law
\begin{equation}
	\bm{\tau} = \bm{M} \left( -\bm{K}\bm{z} -\bm{f}(\bm{\nu}_r) +\dot{r_{\nu}} \right) -\Phi(\bm{\nu}_r )\bm{\hat{\vartheta}}
\end{equation}
And the adaptation law
\begin{equation}
	\bm{\dot{\hat{\vartheta}}} = - \bm{\Gamma} \Phi(\bm{\nu}_r )^T \bm{M}^{-T} \bm{z}
\end{equation}
Will makes the Lyapunov function derivative become
\begin{equation}
	\dot{V} = -\bm{z}^T \bm{K} \bm{z}
\end{equation}
Which is negative semi-definite if $\bm{K}$ is positive definite.

Barbalat’s lemma

\section{Adaptive Position Control}

\subsection{The system}
As in the previous section the system is
\begin{align}
	\bm{\dot{\nu}}_r & = \bm{f}(\bm{\nu}_r) + \bm{M}^{-1}\bm{\tau} + \bm{M}^{-1}\Phi(\bm{\nu}_r )\bm{\vartheta} \\
	\bm{\dot{\eta}}  & = \bm{R}(\psi)\bm{\nu}_r + \bm{\nu}_c
\end{align}
where the known part $\bm{f}(\bm{\nu}_r)$, the uncertain part $\Phi(\bm{\nu}_r )\bm{\vartheta}$ and the input $\bm{\tau}$
of the system is separated. Considering the water currents as a unknown quantity
\begin{align}
	\bm{\dot{\nu}}_r & = \bm{f}(\bm{\nu}_r) + \bm{M}^{-1}\bm{\tau} + \bm{M}^{-1}\Phi_2 (\bm{\nu}_r )\bm{\vartheta} \\
	\bm{\dot{\eta}}  & = \bm{R}(\psi)\bm{\nu}_r + \Phi_1\bm{\vartheta}
\end{align}

\subsection{Control objective}
The control objective is to minimize $\tilde{\eta}(t) \triangleq \eta(t) - r_{\eta}(t)$.
That is to minimize the difference (error) between the position of the vessel $\eta(t)$ and the reference position $r_{\eta}(t)$.
A new variable $\bm{z}$ is introduced to describe this difference.

\begin{align}
	\bm{z}_1       & \triangleq \eta - r_{\eta}                                                                          \\
	\bm{\dot{z}}_1 & = \bm{\dot{\eta}} - \dot{r}_{\eta} = \bm{R}(\psi)\bm{\nu}_r + \Phi_1\bm{\vartheta} - \dot{r}_{\eta}
\end{align}
Since $\bm{\dot{z}}_1 $ does not contain the input $\bm{\tau}$ it is necessary to consider the velocity $\bm{\nu}_r $ as a virtual input $\alpha$.
This is done in order that backstep though the system equations until we find the real input $\bm{\tau}$. For now $\bm{\nu}_r $ is considered a virtual input,
that is, a state the can be controlled to act as the input of a subsystem.

\subsection{Step 1}

Choosing the Lyapunov function

\begin{equation}
	V \triangleq \frac{1}{2}\bm{z}_1^T\bm{z}_1 + \frac{1}{2}\bm{\tilde{\vartheta}}^T\bm{\Gamma}^{-1}\bm{\tilde{\vartheta}}
\end{equation}
Which has the derivative

\begin{align}
	\dot{V} & = \bm{z}_1^T\dot{\bm{z}}_1
	+ \bm{\tilde{\vartheta}} ^T\bm{\Gamma}^{-1}\bm{\dot{\tilde{\vartheta}}}                                        \\
	        & = \bm{z}_1^T\left( \bm{R}(\psi)\alpha + \Phi_1\bm{\vartheta} - \dot{r}_{\eta} \right)
	+ \bm{\tilde{\vartheta}} ^T\bm{\Gamma}^{-1}\bm{\dot{\hat{\vartheta}}}                                          \\
	        & = \bm{z}_1^T\left( \bm{R}(\psi)\alpha + \Phi_1\bm{\hat{\vartheta}} - \dot{r}_{\eta} \right)
	+ \bm{z}_1^T\Phi_1\bm{\tilde{\vartheta}} + \bm{\tilde{\vartheta}} ^T\bm{\Gamma}^{-1}\bm{\dot{\hat{\vartheta}}} \\
	        & = \bm{z}_1^T\left( \bm{R}(\psi)\alpha + \Phi_1\bm{\hat{\vartheta}} - \dot{r}_{\eta} \right)
	+ \bm{\tilde{\vartheta}}^T \left(  \Phi_1^T\bm{z}_1 + \bm{\Gamma}^{-1}\bm{\dot{\hat{\vartheta}}} \right)       \\
\end{align}
It is seen that choosing the control law for the virtual input $\alpha$
\begin{equation}
	\bm{\alpha} = \bm{R}(\psi)^T  -\Phi_1 \bm{\hat{\vartheta}}    +\dot{r_{\eta}} -\bm{K}_1\bm{z}_1
\end{equation}
And the adaptation law
\begin{equation}
	\bm{\dot{\hat{\vartheta}}} = - \bm{\Gamma} \Phi_1^T \bm{z}_1 = - \bm{\Gamma} \tau(\bm{z}_1)
\end{equation}
where $\tau(\bm{z}_1) = \Phi_1^T \bm{z}_1$ is a tuning function, that must not to be confused with the input $\bm{\tau}$.



\subsection{Derivative of a rotation matrix}
Consider the rotation matrix $R(\psi)$

\begin{equation}
	\bm{R}(\psi) = \begin{bmatrix} \cos(\psi) & -\sin(\psi) & 0 \\ \sin(\psi) & \cos(\psi) & 0 \\ 0 & 0 & 1 \end{bmatrix}
\end{equation}
And its derivative
\begin{equation}
	\bm{\dot{R}}(\psi) = \dfrac{d}{d\psi} \bm{R}(\psi) \dot{\psi} =  \dfrac{d}{d\psi} \bm{R}(\psi) r
\end{equation}
\begin{equation}
	\bm{\dot{R}}(\psi) = \begin{bmatrix} -\sin(\psi)r & -\cos(\psi)r & 0 \\ \cos(\psi)r & -\sin(\psi)r & 0 \\ 0 & 0 & 0 \end{bmatrix}
\end{equation}
Which can be written as
\begin{equation}
	\bm{\dot{R}}(\psi) = \bm{S}(r)\bm{R}(\psi)
\end{equation}
With
\begin{equation}
	\bm{S}(r) = \begin{bmatrix} 0 & -r & 0 \\ r & 0 & 0 \\ 0 & 0 & 0 \end{bmatrix}
\end{equation}
This is easily verified by doing the matrix multiplication.

\subsection{Step 2}
Now we must ensure that $\bm{\nu}_r$ takes the value $\alpha$. Thus the control objective
of the second step is

\begin{align}
	\bm{z}_2       & \triangleq \bm{\nu}_r - \bm{\alpha}    \\
	\bm{\dot{z}}_2 & = \bm{\dot{\nu}}_r - \bm{\dot{\alpha}}
\end{align}
\begin{equation}
	\bm{\dot{\alpha}} =  \bm{R}(\psi)^T \bm{S}(r)^T - \Phi_1 \bm{\dot{\hat{\vartheta}}} + \ddot{r}_{\eta} - \bm{K}_1 \bm{\dot{z}}_1
\end{equation}
Augmenting the Lyapunov function with a term for $\bm{z}_2$ we have
\begin{equation}
	V_2 \triangleq V_1 + \frac{1}{2}\bm{z}_2^T\bm{z}_2
\end{equation}
Which has the derivative
\begin{equation}
	\dot{V}_1 = \bm{z}_1^T\left( \bm{R}(\psi)(\bm{z_2} + \bm{\alpha})+ \Phi_1\bm{\hat{\vartheta}} - \dot{r}_{\eta} \right)
	+ \bm{\tilde{\vartheta}}^T \left(  \Phi_1^T\bm{z}_1 + \bm{\Gamma}^{-1}\bm{\dot{\hat{\vartheta}}} \right)
\end{equation}
\begin{equation}
	\dot{V}_1 = \bm{z}_1^T \bm{R}(\psi) \bm{z_2} - \bm{z}_1^T \bm{K}_1 \bm{z}_1 +
	+ \bm{\tilde{\vartheta}}^T \left(  \Phi_1^T\bm{z}_1 + \bm{\Gamma}^{-1}\bm{\dot{\hat{\vartheta}}} \right)
\end{equation}
\begin{align}
	\dot{V} & = \dot{V}_1 + \bm{z}_2^T\dot{\bm{z}}_2                                                                                           \\
	        & = \dot{V}_1 + \bm{z}_2^T \left(  \bm{f}(\bm{\nu}_r) + \bm{M}^{-1}\bm{\tau} + \bm{M}^{-1}\Phi_2 (\bm{\nu}_r )\bm{\vartheta}
	-\bm{R}(\psi)^T \bm{S}(r)^T + \Phi_1 \bm{\dot{\hat{\vartheta}}} - \ddot{r}_{\eta} + \bm{K}_1 \bm{\dot{z}}_1  \right)                       \\
	        & = \dot{V}_1 + \bm{z}_2^T \left(  \bm{f}(\bm{\nu}_r) + \bm{M}^{-1}\bm{\tau} + \bm{M}^{-1}\Phi_2 (\bm{\nu}_r )\bm{\hat{\vartheta}}
	-\bm{R}(\psi)^T \bm{S}(r)^T + \Phi_1 \bm{\dot{\hat{\vartheta}}} - \ddot{r}_{\eta} + \bm{K}_1 \bm{\dot{z}}_1  \right)     +
	\bm{z}_2^T \bm{M}^{-1}\Phi_2 (\bm{\nu}_r )\bm{\tilde{\vartheta}}
	\\
\end{align}
Choosing the control law
\begin{equation}
	\bm{\tau} = -\Phi_2 (\bm{\nu}_r )\bm{\hat{\vartheta}} + M \left( - \bm{K}_2 \bm{z}_2 -\bm{f}(\bm{\nu}_r) +\bm{R}(\psi)^T \bm{S}(r)^T - \Phi_1 \bm{\dot{\hat{\vartheta}}}
	+ \ddot{r}_{\eta} - \bm{K}_1 \bm{\dot{z}}_1    -\bm{z}_1^T \bm{R}(\psi) \bm{z_2}   \right)
\end{equation}

\begin{equation}
	\dot{V} =  - \bm{z}_1^T \bm{K}_1 \bm{z}_1 -  \bm{z}_2^T \bm{K}_2 \bm{z}_2 + \bm{\tilde{\vartheta}}^T \left(  \Phi_1^T\bm{z}_1 +  \Phi_2 (\bm{\nu}_r )^T\bm{M}^{-1}\bm{z}_2 + \bm{\Gamma}^{-1}\bm{\dot{\hat{\vartheta}}} \right)
\end{equation}

The adaptation law
\begin{equation}
	\bm{\dot{\hat{\vartheta}}} = - \bm{\Gamma} \left( \Phi_1^T\bm{z}_1 +  \Phi_2 (\bm{\nu}_r )^T\bm{M}^{-1}\bm{z}_2 \right)
\end{equation}


\end{document}
