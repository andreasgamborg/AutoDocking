\documentclass[12pt,a4]{article}
\usepackage[english]{babel}
\usepackage[utf8]{inputenc}
\usepackage[T1]{fontenc}
\usepackage{geometry}
\usepackage{float}
\geometry{
	a4paper,
	left=20mm,
	right=20mm,
	top=20mm,
	bottom=20mm,
}
% Useful packages
\usepackage{amsmath}
\usepackage{ amssymb }
\usepackage{bm}
\usepackage{graphicx}
\usepackage[colorlinks=true, allcolors=blue]{hyperref}

\begin{document}
\section{The system}
The vessel is modelled by
\begin{align}
	\bm{M}_{RB}\bm{\dot{\nu}} + \bm{M}_{A}\bm{\dot{\nu}_r} + \bm{C}_{RB}(\bm{\nu})\bm{\nu} + \bm{C}_{A}(\bm{\nu}_r)\bm{\nu}_r
	+ \bm{D}(\bm{\nu}_r)\bm{\nu}_r +\bm{G}\bm{\eta} & = \bm{\tau} + \bm{w}(t) \\
	\bm{\dot{\eta}}                                 & = \bm{J}(\eta)\bm{\nu}
\end{align}
where
\begin{align}
	\bm{M}_{RB} & = \begin{bmatrix}
		m & 0 & 0     \\
		0 & m & 0     \\
		0 & 0 & I_{z}
	\end{bmatrix} &
	\bm{M}_{A}  & = \begin{bmatrix}
		-X_{\dot{u}} & 0            & 0            \\
		0            & -Y_{\dot{v}} & -Y_{\dot{r}} \\
		0            & -N_{\dot{v}} & -N_{\dot{r}}
	\end{bmatrix}
\end{align}


\begin{align}
	\bm{C}_{RB} & = \begin{bmatrix}
		0           & 0   & -m(x_g r +v) \\
		0           & 0   & mu           \\
		m(x_g r +v) & -mu & 0
	\end{bmatrix} &
	\bm{C}_{A}  & = \begin{bmatrix}
		0                                                        & 0               & Y_{\dot{v}} v_r + \frac{1}{2}(N_{\dot{v}} Y_{\dot{r}})r \\
		0                                                        & 0               & -X_{\dot{u}} u_r                                        \\
		-Y_{\dot{v}} v_r + \frac{1}{2}(N_{\dot{v}} Y_{\dot{r}})r & X_{\dot{u}} u_r & 0
	\end{bmatrix}
\end{align}

\begin{align}
	\bm{D}_{L}  & = \begin{bmatrix}
		-X_u & 0    & 0    \\
		0    & -Y_v & -Y_r \\
		0    & -N_v & -N_r
	\end{bmatrix} &
	\bm{D}_{NL} & = \begin{bmatrix}
		-X_{|u|u}|u| - X_{uuu}u^2 & 0                           & 0                           \\
		0                         & -Y_{|v|v}|v_r| -Y_{|r|v}|r| & -Y_{|v|r}|v_r| -Y_{|r|r}|r| \\
		0                         & -N_{|v|v}|v_r| -N_{|r|v}|r| & -N_{|v|r}|v_r| -N_{|r|r}|r|
	\end{bmatrix}
\end{align}

Using property 8.1 of [Fossen] we can define the dynamics using only the water relative velocity $\bm{\nu}_r$
\begin{align}
	\label{eq:SysEq} \bm{M}\bm{\dot{\nu}}_r + \bm{C}(\bm{\nu}_r)\bm{\nu}_r +\bm{D}(\bm{\nu}_r)\bm{\nu}_r & = \bm{\tau}                           \\
	\bm{\dot{\eta}}                                                                                      & = \bm{J}(\eta)\bm{\nu}_r + \bm{\nu}_c
\end{align}
Where $\bm{\nu}_c$ is the velocity of the water current in global coordinates. And
\begin{align}
	\bm{M}             & = \bm{M}_{RB} + \bm{M}_{A}                         \\
	\bm{C}(\bm{\nu}_r) & = \bm{C}_{RB}(\bm{\nu}_r) + \bm{C}_{A}(\bm{\nu}_r)
\end{align}
Additionally $\bm{G}\bm{\eta} = \bm{0}$ as there exists no restoring forces in surge, sway and yaw.
Isolating the acceleration $\bm{\dot{\nu}}_r$ in (\ref{eq:SysEq}) we have
\begin{align}
	\bm{\dot{\nu}}_r & = \bm{M}^{-1} (\bm{\tau} -\bm{C}(\bm{\nu}_r)\bm{\nu}_r -\bm{D}(\bm{\nu}_r)\bm{\nu}_r ) \\
	\bm{\dot{\eta}}  & = \bm{J}(\eta)\bm{\nu}_r + \bm{\nu}_c
\end{align}
Assuming that $\bm{M}$ and $\bm{C}$ as well as the linear damping coefficients $X_u$ and $N_r$ are known. The system equation can be split into a known part and a unknown or uncertain part. This can be written as
\begin{align}
	\bm{\dot{\nu}}_r & = \bm{M}^{-1} (\bm{\tau} -\bm{C}(\bm{\nu}_r)\bm{\nu}_r + \bm{d} \bm{\nu}_r + \Phi(\bm{\nu}_r )\bm{\vartheta}) \\
	\bm{\dot{\eta}}  & = \bm{J}(\eta)\bm{\nu}_r + \bm{\nu}_c
\end{align}
Where
\begin{equation}
	\bm{d} = \begin{bmatrix}
		X_u & 0 & 0   \\
		0   & 0 & 0   \\
		0   & 0 & N_r
	\end{bmatrix}
\end{equation}
And the uncertain part is described by
\begin{align}
	\Phi(\bm{\nu}_r)^T & = \begin{bmatrix}
		|u|u & 0 & 0 \\ u^3 & 0 & 0 \\ 0& v &0 \\ 0& r &0 \\ 0&|v|v&0  \\ 0&|r|v&0  \\ 0&|v|r&0  \\ 0&|r|r&0  \\ 0&0& v \\ 0&0& |v|v \\ 0&0&|r|v \\ 0&0&|v|r \\ 0&0&|r|r
	\end{bmatrix}
	                   &
	\bm{\vartheta}     & = \begin{bmatrix}
		X_{|u|u} \\ X_{uuu} \\ Y_v \\ Y_r \\ Y_{|v|v} \\ Y_{|r|v} \\ Y_{|v|r} \\ Y_{|r|r} \\ N_v \\ N_{|v|v} \\ N_{|r|v} \\ N_{|v|r} \\ N_{|r|r}
	\end{bmatrix}
\end{align}
Lastly the known part is collected in a function $\bm{f}(\bm{\nu}_r)$.
\begin{align}
	\bm{\dot{\nu}}_r & = \bm{M}^{-1} (\bm{d} \bm{\nu}_r -\bm{C}(\bm{\nu}_r)\bm{\nu}_r) + \bm{M}^{-1} (\bm{\tau} + \Phi(\bm{\nu}_r )\bm{\vartheta}) \\
	                 & =  \bm{f}(\bm{\nu}_r) + \bm{M}^{-1} (\bm{\tau} + \Phi(\bm{\nu}_r )\bm{\vartheta})                                           \\
\end{align}
The system is now in a form where the known part $\bm{f}(\bm{\nu}_r)$, the uncertain part $\Phi(\bm{\nu}_r )\bm{\vartheta}$ and the input $\bm{\tau}$
are clear to distinguish from one another.

\section{Adaptive Control}
\subsection{Control Design}
The control objective is to minimize $\tilde{\nu}(t) \triangleq \nu(t) - r_{\nu}(t)$.
That is to minimize the difference (error) between the velocity of the vessel $\nu(t)$ and the reference velocity $r_{\nu}(t)$.
A new variable $\bm{z}$ is introduced to describe this difference.

\begin{equation}
	\bm{z} \triangleq \nu - r_{\nu}
\end{equation}
Choosing the Lyapunov function

\begin{equation}
	V \triangleq \frac{1}{2}\bm{z}^T\bm{z} + \frac{1}{2}\bm{\tilde{\vartheta}}^T\bm{\Gamma}^{-1}\bm{\tilde{\vartheta}}
\end{equation}
Which has the derivative

\begin{align}
	\dot{V} & = \bm{z}^T\dot{\bm{z}}
	+ \bm{\tilde{\vartheta}} ^T\bm{\Gamma}^{-1}\bm{\dot{\tilde{\vartheta}}}                                                                         \\
	        & = \bm{z}^T\left(\bm{\dot{\nu}}_r  - \dot{r_{\nu}}\right)
	+ \bm{\tilde{\vartheta}} ^T\bm{\Gamma}^{-1}\bm{\dot{\hat{\vartheta}}}                                                                           \\
	        & = \bm{z}^T\left(\bm{f}(\bm{\nu}_r) + \bm{M}^{-1}\left(\bm{\tau} + \Phi(\bm{\nu}_r )\bm{\vartheta}\right)  - \dot{r_{\nu}}\right)
	+ \bm{\tilde{\vartheta}} ^T\bm{\Gamma}^{-1}\bm{\dot{\hat{\vartheta}}}                                                                           \\
	        & = \bm{z}^T\left(\bm{f}(\bm{\nu}_r) + \bm{M}^{-1}\left(\bm{\tau} + \Phi(\bm{\nu}_r )\bm{\hat{\vartheta}}\right) - \dot{r_{\nu}}\right)
	+ \bm{z}^T\bm{M}^{-1} \Phi(\bm{\nu}_r )\bm{\tilde{\vartheta}}
	+ \bm{\tilde{\vartheta}} ^T\bm{\Gamma}^{-1}\bm{\dot{\hat{\vartheta}}}                                                                           \\
	        & = \bm{z}^T\left(\bm{f}(\bm{\nu}_r) + \bm{M}^{-1}\left(\bm{\tau} + \Phi(\bm{\nu}_r )\bm{\hat{\vartheta}}\right) - \dot{r_{\nu}}\right)
	+ \bm{\tilde{\vartheta}}^T \left(\bm{\Gamma}^{-1}\bm{\dot{\hat{\vartheta}}} +\Phi(\bm{\nu}_r )^T \bm{M}^{-T} \bm{z} \right)                     \\
\end{align}
It is seen that choosing the control law
\begin{equation}
	\bm{\tau} = \bm{M} \left( -\bm{K}\bm{z} -\bm{f}(\bm{\nu}_r) +\dot{r_{\nu}} \right) -\Phi(\bm{\nu}_r )\bm{\hat{\vartheta}}
\end{equation}
And the adaptation law
\begin{equation}
	\bm{\dot{\hat{\vartheta}}} = - \bm{\Gamma} \Phi(\bm{\nu}_r )^T \bm{M}^{-T} \bm{z}
\end{equation}
Will makes the Lyapunov function derivative become
\begin{equation}
	\dot{V} = -\bm{z}^T \bm{K} \bm{z}
\end{equation}
Which is negative definite if $\bm{K}$ is positive definite.
\end{document}
