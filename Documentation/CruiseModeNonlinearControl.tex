\documentclass[12pt,a4]{article}
\usepackage[english]{babel}
\usepackage[utf8]{inputenc}
\usepackage[T1]{fontenc}
\usepackage{geometry}
\usepackage{float}
\geometry{
	a4paper,
	left=20mm,
	right=20mm,
	top=20mm,
	bottom=20mm,
}
% Useful packages
\usepackage{amsmath}
\usepackage{ amssymb }
\usepackage{bm}
\usepackage{graphicx}
\usepackage[colorlinks=true, allcolors=blue]{hyperref}

\begin{document}
\section{The system}

The system is describe by
\begin{align*}
	\dot{\phi} & = r                                                     \\
	\dot{r}    & = (r_{1x} + r_{2x})\sin(\xi)(T_{nn} n^2 + T_{nv} V_A n) \\
	\dot{\xi}  & = \tau_\xi
\end{align*}
Assuming constant propeller speed
\begin{align*}
	\dot{\phi} & = r                                \\
	\dot{r}    & = k_r\sin(\xi)(k_{nn} + k_{nv}V_A) \\
	\dot{\xi}  & = \tau_\xi
\end{align*}
Rewriting this
\begin{align*}
	\dot{x_1} & = x_2 + \bm{\phi}_1^T \vartheta         \\
	\dot{x_2} & = k \sin(x_3) + \bm{\phi}_2^T \vartheta \\
	\dot{x_3} & = u
\end{align*}
where $k = k_{r} + k_{nn}$.
The control objective can then by formulated as a reference tracking of $x_1$.

\begin{align*}
	z_1 & \triangleq x_1 - r                                 \\
	z_2 & \triangleq x_2 - \alpha_1(x_1,\hat{\vartheta})     \\
	z_3 & \triangleq x_3 - \alpha_2(x_1,x_2,\hat{\vartheta})
\end{align*}

\subsection{Step 1}
The dynamics of $z_1$ is

\begin{align}
	z_1       & \triangleq x_1 - r                                                        \\
	\dot{z_1} & = \dot{x_1} - \dot{r}                                                     \\
	          & = x_2 + \bm{\phi}_1^T \vartheta - \dot{r}                                 \\
	          & = z_2 + \alpha_1(x_1,\hat{\vartheta}) + \bm{\phi}_1^T \vartheta - \dot{r}
\end{align}

Considering $x_2$ as a virtual input we can find a stabilizing function for $z_1$

\begin{equation}
	\alpha_1 = - k_1 z_1 -  \bm{\phi}_1^T \hat{\vartheta} + \dot{r}
\end{equation}

If $x_2$ takes the value of $\alpha_1$ we have

\begin{equation}
	\dot{z_1} = x_2 + \bm{\phi}_1^T \vartheta - \dot{r} = - k_1 z_1 -  \bm{\phi}_1^T \hat{\vartheta} + \dot{r}+\bm{\phi}_1^T \vartheta - \dot{r} = - k_1 z_1 - \bm{\phi}_1^T \tilde{\vartheta}
\end{equation}
As $\hat{\vartheta} = \vartheta + \tilde{\vartheta}$. We see that if the estimate $\hat{\vartheta}$ is perfect ($\tilde{\vartheta}=0$), then $z_1$ will be GES for $k_1>0$. In practice we will of course never have a perfect estimate.



Choosing the Lyapunov function

\begin{align}
	V_1       & = \frac{1}{2} z_1^2 + \frac{1}{2 \gamma} \tilde{\vartheta}^2                                                       \\
	\dot{V}_1 & = z_1 \dot{z_1}   +   \frac{1}{\gamma}\tilde{\vartheta}\dot{\tilde{\vartheta}}                                     \\
	          & = z_1 (z_2 + \alpha_1(x_1,\hat{\vartheta}) + \bm{\phi}_1^T \vartheta - \dot{r})
	+ \frac{1}{\gamma}\tilde{\vartheta}\dot{\hat{\vartheta}}                                                                       \\
	          & = z_1 (z_2 - k_1 z_1 -  \bm{\phi}_1^T \hat{\vartheta} + \dot{r} + \bm{\phi}_1^T \vartheta - \dot{r})
	+ \frac{1}{\gamma}\tilde{\vartheta}\dot{\hat{\vartheta}}                                                                       \\
	          & = z_1 z_2 - k_1 z_1^2 - z_1\bm{\phi}_1^T \tilde{\vartheta}
	+ \frac{1}{\gamma}\tilde{\vartheta}\dot{\hat{\vartheta}}                                                                       \\
	          & = z_1 z_2 - k_1 z_1^2 + \tilde{\vartheta} \left( -z_1\bm{\phi}_1^T + \frac{1}{\gamma}\dot{\hat{\vartheta}} \right)
\end{align}
Where the property $\dot{\tilde{\vartheta}} = \dot{\hat{\vartheta}}$ was used. This is due to $\vartheta$ being constant or slowly varying.
We see that chosing the adaptation law
\begin{align}
	\dot{\hat{\vartheta}} & = \gamma   z_1\bm{\phi}_1^T   \\
	                      & =\gamma(x_1 - r)\bm{\phi}_1^T \\
	                      & = \gamma \tau_1(x_1)
\end{align}
Will make the last terms vanish.

\subsection{Step 2}
The dynamics of $z_2$ is
\begin{align}
	z_2       & \triangleq x_2 - \alpha_1(x_1,\hat{\vartheta})                                                                                             \\
	\dot{z_2} & = \dot{x_2} - \frac{\partial \alpha_1}{\partial x_1}  \dot{x_1} - \frac{\partial \alpha_1}{\partial \hat{\vartheta}} \dot{\hat{\vartheta}} \\
	          & = k v + \bm{\phi}_2^T \vartheta + k_1 \dot{x_1}+ \phi_1^T \dot{\hat{\vartheta}}
\end{align}

\begin{equation}
	v = \frac{- \bm{\phi}_2^T \hat{\vartheta} - k_1 \dot{x_1} - \phi_1^T\dot{\hat{\vartheta}} - k_2 z_2}{k}
\end{equation}


\begin{equation}
	\dot{z_2} = - k_2 z_2 - \bm{\phi}_2^T \tilde{\vartheta}
\end{equation}



\begin{align}
	\dot{z_2} & = k v + \bm{\phi}_2^T \vartheta + k_1 \dot{x_1}+ \phi_1^T \dot{\hat{\vartheta}}                                                                                        \\
	          & = - \bm{\phi}_2^T \hat{\vartheta} - k_1 \dot{x_1} - \phi_1^T\dot{\hat{\vartheta}} - k_2 z_2 + \bm{\phi}_2^T \vartheta + k_1 \dot{x_1} + \phi_1^T \dot{\hat{\vartheta}} \\
	          & = - \bm{\phi}_2^T \hat{\vartheta}  - \phi_1^T\dot{\hat{\vartheta}} - k_2 z_2 + \bm{\phi}_2^T \vartheta + \phi_1^T \dot{\hat{\vartheta}}                                \\
\end{align}

Choosing the Lyapunov function

\begin{align}
	V_2       & = V_1 + \frac{1}{2} z_2^2                                                                                     \\
	\dot{V}_2 & = z_1 z_2 - k_1 z_1^2 + \tilde{\vartheta} \left( -\tau_1(x_1) + \frac{1}{\gamma}\dot{\hat{\vartheta}} \right)
	+ z_2 \left(  k v + \bm{\phi}_2^T \vartheta + k_1 \dot{x_1}+ \phi_1^T \dot{\hat{\vartheta}} \right)
\end{align}

\begin{equation}
	v = \frac{-k_2 z_2 - z_1 - \bm{\phi}_2^T \hat{\vartheta} -  k_1 \dot{x_1} -\phi_1^T \dot{\hat{\vartheta}}}{k}
\end{equation}

\begin{equation}
	\dot{V}_2  = - k_1 z_1^2 - k_2 z_2^2 + \tilde{\vartheta} \left( -\tau_1(x_1) + \frac{1}{\gamma}\dot{\hat{\vartheta}} - z_2\bm{\phi}_2^T	\right)
\end{equation}

We see that chosing the adaptation law
\begin{align}
	\dot{\hat{\vartheta}} & = \gamma   ( z_1\bm{\phi}_1^T + z_2\bm{\phi}_2^T ) \\
\end{align}
Will make the last terms vanish.


\begin{align}
	v & = \frac{-k_2 (x_2-(- k_1 z_1 -  \bm{\phi}_1^T \hat{\vartheta} + \dot{r}))- (x_1-r)- \bm{\phi}_2^T \hat{\vartheta} -  k_1 \dot{x_1} -\phi_1^T \dot{\hat{\vartheta}}}{k}                  \\
	v & = \frac{-k_2 x_2 - k_2 k_1 z_1 -  k_2\bm{\phi}_1^T \hat{\vartheta} + k_2\dot{r} -x_1 +r -\bm{\phi}_2^T \hat{\vartheta} -  k_1\dot{x_1} -\phi_1^T \dot{\hat{\vartheta}}}{k}              \\
	v & = \frac{-k_2 x_2 - k_2 k_1 x_1 + k_2 k_1 r -  k_2\bm{\phi}_1^T \hat{\vartheta} + k_2\dot{r} -x_1 + r -\bm{\phi}_2^T \hat{\vartheta} -  k_1\dot{x_1} -\phi_1^T \dot{\hat{\vartheta}}}{k} \\
	v & = \frac{-k_2 x_2 + (r - x_1)(k_2 k_1 + 1) + k_2\dot{r}  -  k_1\dot{x_1} - \phi_1^T( k_2\hat{\vartheta} + \dot{\hat{\vartheta}} ) -\bm{\phi}_2^T \hat{\vartheta}  }{k}
\end{align}

\begin{align}
	-1 & \leq v \leq 1                                                                                                         \\
	-1 & \leq \frac{-k_2 z_2 - z_1 - \bm{\phi}_2^T \hat{\vartheta} -  k_1 \dot{x_1} -\phi_1^T \dot{\hat{\vartheta}}}{k} \leq 1 \\
	-k & \leq -k_2 z_2 - z_1 - \bm{\phi}_2^T \hat{\vartheta} -  k_1 \dot{x_1} -\phi_1^T \dot{\hat{\vartheta}}\leq k
\end{align}

\end{document}
