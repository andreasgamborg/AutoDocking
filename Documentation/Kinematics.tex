\documentclass[12pt,a4]{article}
\usepackage[english]{babel}
\usepackage[utf8]{inputenc}
\usepackage[T1]{fontenc}
\usepackage{geometry}
\usepackage{float}
\geometry{
	a4paper,
	left=20mm,
	right=20mm,
	top=20mm,
	bottom=20mm,
}
% Useful packages
\usepackage{amsmath}
\usepackage{graphicx}
\usepackage[colorlinks=true, allcolors=blue]{hyperref}

\begin{document}

\section{Kinematics}
\begin{equation}
	M \dot{\nu} + C(\nu)\nu + D(\nu)\nu + g(\eta) + g_0 = \tau
\end{equation}
\begin{align*}
	M  = M_{RB} + M_A \\
	C  = C_{RB} + C_A
\end{align*}

\begin{equation}
	\tau = \tau_{hydrodynamics}+\tau_{hydrostatics}+\tau_{wind}+\tau_{wave}+\tau_{control}
\end{equation}


\subsection{Control}
\begin{equation}
	\tau_{control} = \begin{bmatrix}
		\tau_{control,linear} \\
		\tau_{control,torque}
	\end{bmatrix}
\end{equation}

Each propeller can rotate with an angle $\xi_p$. The thrust of each propeller can be describe as a vector by spilting it into components that align with the ship coordinate frame
\begin{equation}
	T_p = \begin{bmatrix} \cos(\xi_p)\\ \sin(\xi_p)\\ 0 \end{bmatrix} t_p
\end{equation}
where $t_p$ is the magnitude of the thrust and $T_p$ is the thrust vector.
The linear control force is found as the sum of forces from each propeller p.
\begin{equation}
	\tau_{control,linear} = \sum^P T_p
\end{equation}

\begin{equation}
	\tau_{control,torque} = \sum^P  r_p \times T_p = \sum^P S(r_p) T_p
\end{equation}
where $r_p$ is the position vector of thruster \textit{p} and $S(r_p)$ is the skew-symetric matrix of vector $r_p$.

In the case of two propellers we have
\begin{equation}
	\tau_{control} = \begin{bmatrix} I & I \\ S(r_1) & S(r_2) \end{bmatrix}\begin{bmatrix} T_1 \\ T_2 \end{bmatrix}
\end{equation}
\end{document}
