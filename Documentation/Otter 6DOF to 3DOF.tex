\documentclass[12pt,a4]{article}
\usepackage[english]{babel}
\usepackage[utf8]{inputenc}
\usepackage[T1]{fontenc}
\usepackage{geometry}
\usepackage{float}
\geometry{
	a4paper,
	left=20mm,
	right=20mm,
	top=20mm,
	bottom=20mm,
}
% Useful packages
\usepackage{amsmath}
\usepackage{bm}
\usepackage{graphicx}
\usepackage[colorlinks=true, allcolors=blue]{hyperref}

\begin{document}

\section{Mass- and Coriolis matrices of the 6 DOF model}
In the case of no payload and no current we have

\begin{equation*}
	M_{RB} = \left[\begin{array}{cccccc} 55 & 0 & 0 & 0 & -11 & 0\\ 0 & 55 & 0 & 11 & 0 & 11\\ 0 & 0 & 55 & 0 & -11 & 0\\ 0 & 11 & 0 & 14.6643 & 0 & 4.4000\\ -11 & 0 & -11 & 0 & 22.5500 & 0\\ 0 & 11 & 0 & 4.4000 & 0 & 18.1500 \end{array}\right]
\end{equation*}
\begin{align*}
	 & C_{RB} =                               \\
	 & \left[\begin{array}{cccccc} 0 & -55\,r & 55\,q & -11\,r & -11\,q & -11\,r\\ 55\,r & 0 & -55\,p & 0 & 11\,p-11\,r & 0\\ -55\,q & 55\,p & 0 & 11\,p & 11\,q & 11\,p\\ 11\,r & 0 & -11\,p & 0 & 4.4000\,p+13.7500\,r & -18.1500\,q\\ 11\,q & 11\,r-11\,p & -11\,q & -4.4000\,p-13.7500\,r & 0 & 10.2643\,p+4.4000\,r\\ 11\,r & 0 & -11\,p & 18.1500\,q & -10.2643\,p-4.4000\,r & 0 \end{array}\right]
\end{align*}
\begin{equation*}
	M_{A} = \left[\begin{array}{cccccc} 5.5000 & 0 & 0 & 0 & 0 & 0\\ 0 & 82.5000 & 0 & 0 & 0 & 0\\ 0 & 0 & 55 & 0 & 0 & 0\\ 0 & 0 & 0 & 2.4929 & 0 & 0\\ 0 & 0 & 0 & 0 & 14.5200 & 0\\ 0 & 0 & 0 & 0 & 0 & 27.1150 \end{array}\right]
\end{equation*}
\begin{equation*}
	C_{A} = \left[\begin{array}{cccccc} 0 & 0 & 0 & 0 & 55\,w & -82.5000\,v\\ 0 & 0 & 0 & -55\,w & 0 & 5.5000\,u\\ 0 & 0 & 0 & 82.5000\,v & -5.5000\,u & 0\\ 0 & 55\,w & -82.5000\,v & 0 & 27.1150\,r & -14.5200\,q\\ -55\,w & 0 & 5.5000\,u & -27.1150\,r & 0 & 2.4929\,p\\ 0 & 0 & 0 & 14.5200\,q & -2.4929\,p & 0 \end{array}\right]
\end{equation*}
\begin{equation*}
	D = \left[\begin{array}{cccccc} 77.5544\,u & 0 & 0 & 0 & 0 & 0\\ 0 & 0 & 0 & 0 & 0 & 0\\ 0 & 0 & 546.4805\,w & 0 & 0 & 0\\ 0 & 0 & 0 & 54.3823\,p & 0 & 0\\ 0 & 0 & 0 & 0 & 246.0496\,q & 0\\ 0 & 0 & 0 & 0 & 0 & r\,\left(452.6500\,\left|r\right|+45.2650\right) \end{array}\right]
\end{equation*}


\section{Sytem equations}
\begin{equation}
	\dot{\bm{\nu}} = (\bm{M}_{RB}+\bm{M}_{A})^{-1}(-\bm{C}_{RB}-\bm{C}_{A}-\bm{D})\bm{\nu}
\end{equation}
\begin{equation}
	\dot{\bm{\nu}} = f(\bm{\nu}) = f(u, v, w, p, q, r)
\end{equation}
\begin{equation}
	\dot{\bm{\nu}} = \begin{bmatrix}\dot{u}&\dot{v}&\dot{w}&\dot{p}&\dot{q}&\dot{r}\end{bmatrix}^T
\end{equation}
\begin{multline}
	\dot{u} = 0.0177\,p^2+0.3371\,p\,r-0.0147\,v\,p-1.1303\,q^2-0.0265\,q\,u-1.8664\,q\,w+\\
	0.1690\,r^2+2.3477\,v\,r-1.3574\,u^2+0.2649\,u\,w-0.2925\,w^2
\end{multline}
\begin{multline}
	\dot{v} = 0.6506\,r^2\,\left|r\right|-0.0364\,p\,q+0.1167\,q\,r+0.7867\,p\,w-\\
	0.4028\,r\,u-0.1266\,v\,w+0.2504\,p^2+0.0651\,r^2
\end{multline}
\begin{multline}
	\dot{w} = -0.0903\,p^2-0.0146\,p\,r-1.2581\,v\,p-0.8217\,q^2+0.5354\,q\,u-0.0265\,q\,w-\\
	0.0071\,r^2+0.0412\,v\,r-0.0415\,u^2+0.1457\,u\,w-5.1289\,w^2
\end{multline}
\begin{multline}
	\dot{p} = 2.2439\,r^2\,\left|r\right|-0.1257\,p\,q-0.5847\,q\,r+0.1266\,p\,w-\\
	0.3545\,r\,u+1.7190\,v\,w-3.3993\,p^2+0.2244\,r^2
\end{multline}
\begin{multline}
	\dot{q} = 0.0971\,p^2+0.8539\,p\,r-0.0810\,v\,p-7.2166\,q^2-0.1457\,q\,u-0.2649\,q\,w-\\
	0.0707\,r^2+0.4121\,v\,r-0.4151\,u^2+1.4572\,u\,w-1.6087\,w^2
\end{multline}
\begin{multline}
	\dot{r} = 0.1257\,q\,r-0.4188\,p\,q-10.3762\,r^2\,\left|r\right|+0.0395\,p\,w-\\
	0.1107\,r\,u-0.1363\,v\,w+0.2696\,p^2-1.0376\,r^2
\end{multline}

\begin{multline}
	A = \dfrac{df}{d\bm{\nu}}\\
	\left[\begin{array}{ccc} 0.2649\,w-2.7147\,u-0.0265\,q & 2.3477\,r-0.0147\,p & 0.3371\,p+0.3379\,r+2.3477\,v\\ -0.4028\,r & -0.1266\,w & 0.1167\,q+0.0651\,r-0.4028\,u+r\,\left(0.6506\,\left|r\right|+0.6506\,r\,\mathrm{sign}\left(r\right)+0.0651\right)+0.6506\,r\,\left|r\right|\\ -0.1107\,r & -0.1363\,w & 0.1257\,q-1.0376\,r-0.1107\,u-r\,\left(10.3762\,\left|r\right|+10.3762\,r\,\mathrm{sign}\left(r\right)+1.0376\right)-10.3762\,r\,\left|r\right| \end{array}\right]
\end{multline}

\section{Simple state truncation}
\begin{equation*}
	\dot{\nu} = A \nu + B \tau
\end{equation*}
\begin{align*}
	 & A =  -(M_{RB} + M_A)^{-1}(C_{RB}+C_A+D)= \\
	 & \left[\begin{array}{ccc} 0.2944\,w-1.3574\,u-0.0294\,q & 0.0294\,p+0.9037\,r & 0.1690\,r-0.0742\,p+1.4439\,v\\ -0.3601\,r & 0.2532\,w & 0.0651\,r-0.1504\,q-0.0427\,u+0.6506\,r\,\left|r\right|\\ -0.1186\,r & 0.2726\,w & 0.0079\,u-1.0376\,r-0.1620\,q-10.3762\,r\,\left|r\right| \end{array}\right]
\end{align*}
\begin{align*}
	 & B = (M_{RB} + M_A)^{-1}=                \\
	 & \left[\begin{array}{ccc} 0.0175 & 0 & 0\\ 0 & 0.0078 & -0.0014\\ 0 & -0.0014 & 0.0229 \end{array}\right]
\end{align*}
where the heave, roll and pitch dimensions has been removed from \textit{A} and \textit{B}

\subsection{Reduction}

Assiming that

\begin{align*}
	w & = 0 \\
	p & = 0 \\
	q & = 0
\end{align*}
The system equations simplify to

\begin{equation}
	A(u,v,r) = \left[\begin{array}{ccc} -1.3574\,u & 0.9037\,r & 0.1690\,r+1.4439\,v\\ -0.3601\,r & 0 & 0.0651\,r-0.0427\,u+0.6506\,r\,\left|r\right|\\ -0.1186\,r & 0 & 0.0079\,u-1.0376\,r-10.3762\,r\,\left|r\right| \end{array}\right]
\end{equation}
\begin{equation}
	B = \left[\begin{array}{ccc} 0.0175 & 0 & 0\\ 0 & 0.0078 & -0.0014\\ 0 & -0.0014 & 0.0229 \end{array}\right]
\end{equation}


\section{Selection matrix approch}

\begin{align}
	C^*_{RB} & = UM_{RB}L
	         &
	C^*_{A}  & = UM_AL
\end{align}
Where \textit{U} is the surge of the vessel at the linarization point and L is a selection matrix seen in HMCHMC (3.63).
For this lineariztion $U=3$
\begin{align}
	M_{RB} & = \left[\begin{array}{ccc} 55 & 0 & 0\\ 0 & 55 & 11\\ 0 & 11 & 18.1500 \end{array}\right]
	       &
	M_{A}  & = \left[\begin{array}{ccc} 5.5000 & 0 & 0\\ 0 & 82.5000 & 0\\ 0 & 0 & 27.1150 \end{array}\right]
\end{align}

\begin{align}
	C^*_{RB} & = \left[\begin{array}{ccc} 0 & 0 & 0\\ 0 & 0 & 165\\ 0 & 0 & 33 \end{array}\right]
	         &
	C^*_{A}  & = \left[\begin{array}{ccc} 0 & 0 & 0\\ 0 & 0 & 247.5000\\ 0 & 0 & 0 \end{array}\right]
\end{align}
\begin{equation}
	D^* = \left[\begin{array}{ccc} 232.6633 & 0 & 0\\ 0 & 0 & 0\\ 0 & 0 & 497.9150 \end{array}\right]
\end{equation}
$D*$ has been linearized in yaw around the point $r = 1$

\begin{equation}
	\dot{\nu} = A \nu + B \tau
\end{equation}

\begin{align}
	M & = M_{RB} + M_A           \\
	N & = C^*_{RB} + C^*_{A} + D \\
	A & = -M^{-1}N               \\
	B & = M^{-1}
\end{align}

\begin{align}
	A & = \left[\begin{array}{ccc} -3.8457 & 0 & 0\\ 0 & 0 & -2.1026\\ 0 & 0 & -11.2181 \end{array}\right]
	  &
	B & = \left[\begin{array}{ccc} 0.0165 & 0 & 0\\ 0 & 0.0074 & -0.0018\\ 0 & -0.0018 & 0.0225 \end{array}\right]
\end{align}



\end{document}
