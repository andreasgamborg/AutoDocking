\documentclass[12pt,a4]{article}
\usepackage[english]{babel}
\usepackage[utf8]{inputenc}
\usepackage[T1]{fontenc}
\usepackage{geometry}
\usepackage{float}
\geometry{
	a4paper,
	left=20mm,
	right=20mm,
	top=20mm,
	bottom=20mm,
}
% Useful packages
\usepackage{amsmath}
\usepackage{bm}
\usepackage{graphicx}
\usepackage[colorlinks=true, allcolors=blue]{hyperref}

\begin{document}

\section{Kinematics}
\begin{align}
	\bm{M}_{RB}\bm{\dot{\nu}} + \bm{M}_{A}\bm{\dot{\nu_r}} + \bm{C}_{A}(\bm{\nu})\bm{\nu} + \bm{C}_{RB}(\bm{\nu}_r)\bm{\nu}_r + \bm{D}(\bm{\nu}_r)\bm{\nu}_r & = \bm{\tau} + \bm{w}(t) \\
	\bm{\dot{\eta}}                                                                                                                                          & = \bm{J}(\eta)\bm{\nu}
\end{align}
\begin{equation}
	\bm{\tau} = \bm{\tau}_{control}
\end{equation}
%\begin{equation}
%	\tau = \tau_{hydrodynamics}+\tau_{hydrostatics}+\tau_{wind}+\tau_{wave}+\tau_{control}
%\end{equation}

\section{Thruster model}

\begin{equation}
	\bm{\tau}_{control} = \begin{bmatrix}
		\bm{\tau}_{control,linear} \\
		\bm{\tau}_{control,torque}
	\end{bmatrix}
\end{equation}

Each propeller can rotate with an angle $\xi_p$. The thrust of each propeller can be describe as a vector by spilting it into components that align with the ship coordinate frame
\begin{equation}
	t_p = T_{nn}n_p^2+T_{nv}V_A n_p
\end{equation}
\begin{equation}
	\bm{T}_p = \begin{bmatrix} \cos(\xi_p)\\ \sin(\xi_p)\\ 0 \end{bmatrix} t_p
\end{equation}
where $t_p$ is the magnitude of the thrust and $\bm{T}_p$ is the thrust vector.
The linear control force is found as the sum of forces from each propeller p.
\begin{equation}
	\bm{\tau}_{control,linear} = \sum^P \bm{T}_p
\end{equation}

\begin{equation}
	\bm{\tau}_{control,torque} = \sum^P  \bm{r}_p \times \bm{T}_p = \sum^P S(\bm{r}_p) \bm{T}_p
\end{equation}
where $\bm{r}_p$ is the position vector of thruster \textit{p} and $S(\bm{r}_p)$ is the skew-symetric matrix of vector $\bm{r}_p$.

In the case of two propellers we have
\begin{equation}
	\bm{\tau}_{control} = \begin{bmatrix} I & I \\ S(\bm{r}_1) & S(\bm{r}_2) \end{bmatrix}\begin{bmatrix} \bm{T}_1 \\ \bm{T}_2 \end{bmatrix}
\end{equation}

\end{document}
